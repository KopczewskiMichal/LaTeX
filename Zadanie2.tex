\documentclass{beamer}
\usetheme{Warsaw}
\usecolortheme{dolphin}
\usepackage[utf8]{inputenc}
\usepackage[MeX]{polski}


\title{Zadanie2}
\author{Michał Kopczewski}
\date{February 2023}

\begin{document}

\maketitle

\begin{frame}[label=Karol]
\begin{figure}[htbp]
\centering
\includegraphics[width=0.9\textwidth]{foto1.jpg}
\caption{Najlepszy teleturniej}
\end{figure}

\end{frame}
\begin{frame}[label=Logo]

\begin{figure}[htp]
\begin{minipage}{0.33\textwidth}
\centering
\includegraphics[width=\textwidth]{foto2.jpg}
\end{minipage}
\begin{minipage}{0.66\textwidth}
\centering
\textit{Familiada} – polski teleturniej emitowany na antenie TVP2 od 17 września 1994 roku, oparty na amerykańskim formacie Family Feud. 
Jego prowadzącym od samego początku jest Karol Strasburger, choć według wstępnych planów funkcję tę miał pełnić Paweł Wawrzecki. Program 
nagrywany jest w studiu telewizyjnym w Warszawie przy ul. Inżynierskiej 4, w budynku dawnego kina „Syrena”.
\end{minipage}
\end{figure}

\end{frame}
\begin{frame}

\frametitle{}
\small{W programie uczestniczą przedstawiciele dwóch drużyn, z których każda liczy 5 osób. Początkowo zespoły mogły składać się 
wyłącznie z członków jednej rodziny, lecz od 2007 roku zespoły mogą tworzyć także grupy znajomych. Każda drużyna ma wyznaczonego 
kapitana – głowę drużyny (bądź głowę rodziny). Gra polega na udzielaniu takich odpowiedzi, na które wskazało wcześniej najwięcej 
ankietowanych osób. Teleturniej dzieli się na dwa etapy: pierwszy, który trwa tak długo, aż jedna z drużyn zdobędzie łącznie 
przynajmniej trzysta punktów (składa się zatem z dowolnej liczby rund; w praktyce najczęściej pięciu, czasem czterech lub sześciu) oraz 
finał. Drużyna, która dostaje się do finału, bierze udział w następnym programie, przy czym może wystąpić w maksymalnie trzech kolejnych 
odcinkach. Każda drużyna ma swój kolor – czerwony lub niebieski; po czerwonej części studia staje drużyna, która w poprzednim odcinku 
weszła do finału, po niebieskiej zawsze drużyna debiutująca. W przypadku, gdy do gry przystępują dwie nowe drużyny, o zajmowanym 
stanowisku decyduje losowanie.}

\end{frame}
\begin{frame}[label=tabela]

\resizebox{\textwidth}{!}{
\begin{center}
    \begin{tabular}{ | c | c | }
\hline
\textbf{Przybliżony okres} & \textbf{Pora emisji}\\
\hline
sezon 1994/1995 & sobota i niedziela 16.30\\
sezon 1995/1996 & sobota i niedziela ok. 15.00\\
wakacje 1996 & piątek ok. 16.30, sobota i niedziela 15.00\\
sezon 1996/1997 & sobota i niedziela 15.\\
sezony 1997/1998 i 1998/1999 & sobota, niedziela i święta ok. 15.00\\
sezony 1999/2000 i 2000/2001 & sobota, niedziela i święta 14.30\\
od września 2001 & sobota, niedziela i święta 14.00[b]\\
\hline
\end{tabular}
\end{center}
}

\end{frame}
\begin{frame}[label=StartLoremu]

\large{\textbf{Tutaj koniec merytoryki, Lorem Ipsum Wita!}}\\
Lorem ipsum dolor sit amet, consectetur adipiscing elit. Sed fringilla libero in leo feugiat, sed gravida purus vestibulum. Donec in 
convallis ante, ut dictum tortor.
\begin{itemize}
    \item Morbi ut facilisis eros, tempus mattis erat. Ut nec nunc mauris.
    \item Suspendisse sollicitudin diam at ligula mattis.
    \item Pellentesque efficitur euismod sagittis.
\end{itemize}
\end{frame}
\begin{frame}{Tekst i fotka}

\begin{figure}[htp]
\begin{minipage}{0.33\textwidth}
\centering
\includegraphics[width=\textwidth]{foto3.jpeg}
\end{minipage}
\begin{minipage}{0.66\textwidth}
\centering
\textit{Lorem Ipsum} – Proin egestas mauris tincidunt ornare pharetra. Curabitur magna tellus, maximus in eleifend vel, auctor id nulla. 
Duis nec lacus malesuada, ullamcorper neque non, tristique libero. Sed sollicitudin interdum elementum. 
\end{minipage}
\end{figure}
    
\end{frame}
\begin{frame}[label=Listaloremów]

\begin{enumerate}
\item Nam suscipit augue ut mollis scelerisque.
\item Praesent purus mi, tempus eget rutrum ac, lacinia sit amet leo.
\item Phasellus porta neque nec mollis aliquam. 
\end{enumerate}
\textsc{Sed lobortis ante eget est vulputate, eu tincidunt ipsum maximus. Quisque velit velit, porta sit amet venenatis vitae, ultricies 
nec ipsum.}

\end{frame}
\begin{frame}[label=TekstLorem]

\begin{figure}[htp]
\begin{minipage}{0.33\textwidth}
\centering
\includegraphics[width=\textwidth]{foto4.jpeg}
\end{minipage}
\begin{minipage}{0.66\textwidth}
\centering
\textit{Lorem Ipsum} – Suspendisse et congue dolor. Vivamus eget rutrum elit. Mauris ut metus et urna ullamcorper pretium mollis in 
neque. Nunc vitae dui ex. Etiam ut consequat mauris. \begin{itemize}
    \item Morbi ut facilisis eros, tempus mattis erat. Ut nec nunc mauris.
    \item Suspendisse sollicitudin diam at ligula mattis.
    \item Pellentesque efficitur euismod sagittis.
\end{itemize}.
\end{minipage}
\end{figure}

\end{frame}
\begin{frame}

\textbf{Odwołania do elementów}
\begin{enumerate}
\item \hyperlink{Karol}{Karol Strasburger}
\item \hyperlink{Tabela}{Tabelka godzin emisji familiady}
\item \hyperlink{StartLoremu}{Tekst Loremu}
\item \hyperlink{Listaloremów}{Lista numerowana}
\item \hyperlink{TekstLorem}{Ostatni obrazek}
\end{enumerate}
\textbf{Bibliografia}
\begin{enumerate}
    \item \hyperlink{https://pl.wikipedia.org/wiki/Familiada}{Wikipedia Familiady} \hyperlink{Odnośnik do foto Strasburgera}{Familiada}
    \item \hyperlink{https://www.lipsum.com/feed/html}{Generator Lorem} \hyperlink{StartLoremu}{Odnośnik do pierszego slajdu tekstem 
Lorem Ipsum}
\end{enumerate}
\end{frame}

\end{document}

