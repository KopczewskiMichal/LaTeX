\documentclass{article}
\usepackage[utf8]{inputenc}
\usepackage[MeX]{polski}
\usepackage{enumerate}
\usepackage{graphicx}
\usepackage{hyperref}

\title{Zadanie1}
\author{Michał Kopczewski}


\begin{document}

\maketitle

\newpage

\begin{abstract}
    Moja praca zaczyna się od tekstu Lorem Ipsum. Jest to teks napisany po łacinie wyłącznie po to aby był bezsensowny i niezrozumiały. 
Następnie zaprezentowałem kilka wzorów na całki, by na koniec przejść do informacji o teleturnieju Familiada.
\end{abstract}

\tableofcontents

\newpage

\section{Lorem Ipsum}
\subsection{Początek rozdziału}
\textbf{Lorem ipsum dolor sit amet}, consectetur adipiscing elit. Donec auctor dignissim orci tincidunt efficitur. Aenean id purus 
accumsan, laoreet lectus vitae, dignissim leo. Curabitur dignissim eros quam. Phasellus erat metus, interdum eget egestas non, sodales 
id quam. Curabitur hendrerit nibh vel laoreet facilisis. Integer diam nisi, tincidunt sit amet pretium sit amet, facilisis ut elit. 
Nulla facilisi. Nullam sit amet quam sit amet arcu interdum auctor in eget odio. Aenean purus orci, vehicula a nibh sed, ultrices 
iaculis ipsum. Nullam nec libero quis lectus consequat tincidunt. Etiam nec urna pellentesque, elementum dui ut, molestie lorem. Nunc 
fermentum, est sed cursus malesuada, turpis magna molestie nulla, eget dictum neque est quis neque. Cras id lacinia ipsum.

\textit{Nunc ut ipsum nec tellus blandit dapibus id porta lorem. Aliquam blandit eu lacus vel fermentum. Sed et odio sit amet dui 
dapibus rutrum. Fusce auctor velit et venenatis auctor.} Cras metus ex, pellentesque non felis at, tristique sollicitudin eros. Proin 
neque eros, blandit vitae ligula nec, sollicitudin aliquam ligula. Morbi enim enim, sagittis at sapien ut, egestas posuere risus. Nunc 
ornare id risus sit amet vulputate. Aliquam mollis, lectus et tincidunt lacinia, tellus nisi interdum ante, eget volutpat arcu ante at 
eros. In vel ipsum quam. Vestibulum ante ipsum primis in faucibus orci luctus et ultrices posuere cubilia curae; Maecenas at velit 
lorem.

Vestibulum sit amet erat magna. Proin sed convallis libero. Nulla leo turpis, aliquam sed ipsum in, hendrerit scelerisque mi. Duis eu 
ligula molestie, vestibulum neque eu, posuere nulla. In id scelerisque dui. Sed dolor tortor, commodo nec orci quis, fringilla feugiat 
tellus. Mauris id magna lobortis, ultricies lacus eu, congue orci. Nam tincidunt finibus finibus. In purus est, hendrerit ac semper nec, 
finibus id odio.

\subsection{Środek rozdziału}
Pellentesque eu lorem commodo erat semper euismod. Sed eget facilisis nibh, a hendrerit dolor. Quisque id ex tristique, blandit lacus 
eget, faucibus leo. Nam luctus maximus arcu vitae condimentum. Mauris ultricies, quam vel sollicitudin tempus, elit justo lobortis 
turpis, nec hendrerit mi augue in magna. Vivamus eros lectus, efficitur sit amet nulla sit amet, consectetur mattis purus. Duis pretium 
fermentum lobortis. Nunc mauris mi, volutpat vitae mattis id, rutrum sed quam. Aliquam in consequat quam.

Nunc eu euismod diam, eget placerat sapien. Praesent blandit, ligula id dignissim viverra, augue mi rutrum libero, sed sodales urna 
ligula in nunc. Praesent id mi eget neque feugiat faucibus et non dolor. Vestibulum enim tellus, \underline{auctor malesuada mi eu, 
mattis lobortis purus.} Nullam a molestie enim. Nunc massa mauris, pulvinar sed massa sed, tincidunt rutrum nunc. Ut facilisis nulla a 
feugiat auctor. Cras vitae lobortis ex. Quisque eget orci vel ipsum convallis placerat. Donec nibh justo, sollicitudin quis quam vitae, 
interdum porttitor odio. Orci varius natoque penatibus et magnis dis parturient montes, nascetur ridiculus mus.

Sed ut justo vestibulum, bibendum nunc et, aliquam eros. Ut euismod viverra ultricies. In sagittis mi eu venenatis mattis. Maecenas nec 
enim turpis. Nulla tempor dictum elit non euismod. In leo orci, congue at ullamcorper et, pharetra eget lacus. Vestibulum mattis leo non 
arcu porta, eu vestibulum velit aliquet. In interdum quis elit nec malesuada. Curabitur eleifend ipsum in nibh sollicitudin scelerisque. 
Morbi ultricies, risus eu aliquam consequat, metus turpis faucibus elit, at pretium neque odio eu ipsum. Maecenas dapibus malesuada 
pharetra. Aliquam egestas sem pulvinar, rutrum orci a, porta sapien. Suspendisse potenti. Aliquam ipsum enim, rutrum sed urna ac, 
fringilla congue magna. Quisque sit amet convallis nibh, nec viverra odio.

Donec egestas velit vel dolor tincidunt, id pulvinar mauris convallis. Nulla fringilla vitae sapien at ultricies. Sed tempus venenatis 
nibh, at dictum ex ultricies quis. In elementum libero lorem, at porttitor eros porta sit amet. Aenean consectetur metus est, nec auctor 
lectus sagittis a. Vestibulum lacinia rutrum odio. Sed varius nulla urna, at accumsan est porttitor quis. Quisque maximus euismod 
lobortis.

Nam egestas maximus dolor, at cursus erat consequat eu. Morbi ultricies pellentesque metus, ac tempus est blandit a. Pellentesque 
habitant morbi tristique senectus et netus et malesuada fames ac turpis egestas. Vivamus nec faucibus elit. Maecenas ultricies metus 
nisl, sed pharetra metus consequat quis. Vivamus auctor urna eget tincidunt vestibulum. Suspendisse vulputate erat sit amet enim dapibus 
fermentum. Sed hendrerit iaculis enim nec ullamcorper.

Suspendisse maximus nibh ut arcu feugiat interdum. Pellentesque habitant morbi tristique senectus et netus et malesuada fames ac turpis 
egestas. Nunc molestie fermentum massa id tempor. In gravida eget nibh cursus porta. Morbi lorem dui, tincidunt eget nulla et, consequat 
condimentum turpis. Suspendisse potenti. Pellentesque ornare vel nisi id luctus. Curabitur consequat nibh tempus lacus fringilla 
hendrerit.


\subsection{Zakończenie rozdziału}
\textsc{Sed ut lacus ultrices, elementum erat eget, ullamcorper augue. \underline{Curabitur sed placerat sem.} In scelerisque odio nunc, 
posuere elementum sem pharetra vel. Lorem ipsum dolor sit amet, consectetur adipiscing elit. Morbi justo nisl, ultricies ut rhoncus 
quis, convallis a tellus. Proin ut quam porttitor tortor auctor dapibus eget blandit mauris. Sed quis enim sollicitudin velit ornare 
consectetur eu id mi. In eu nulla diam. Praesent interdum accumsan congue. Vivamus consequat libero arcu, posuere interdum tellus 
scelerisque quis. Quisque turpis mauris, condimentum ac iaculis varius, vestibulum non metus. Aliquam erat volutpat. Vivamus a dignissim 
justo, vel tristique nunc. Nunc justo massa, venenatis ac mattis a, molestie vel velit. Etiam fringilla est tortor, sit amet mattis arcu 
aliquam ut. Integer egestas ullamcorper urna vitae vestibulum.}

\section{Matematyka}
\subsection{Wzory na całki}
\begin{enumerate}
\item $\int\frac{1}{x}dx = ln|x| + C$
\item $\int x^{n} dx = \frac{1}{n+1} x^{n+1} + C $
\item $\int\frac{dx}{x^{2} - a ^{2}} = \frac{1}{2a}ln|\frac{x-a}{x+a}|$
\item $\int\frac{dx}{\sqrt{x^{2} + q}} = ln|x+\sqrt{x^{2}+q}| + C$
\end{enumerate}

\section{Familiada}
\subsection{Karol Strasburger}
\begin{figure}[htbp]
\centering
\includegraphics[width=0.5\textwidth]{foto1.jpg}
\caption{Najlepszy teleturniej}
\end{figure}

\subsection{O programie}
\begin{figure}[htp]
\begin{minipage}{0.33\textwidth}
\centering
\includegraphics[width=\textwidth]{foto2.jpg}
\end{minipage}
\begin{minipage}{0.66\textwidth}
\centering
\textit{Familiada} – polski teleturniej emitowany na antenie TVP2 od 17 września 1994 roku, oparty na amerykańskim formacie Family Feud. 
Jego prowadzącym od samego początku jest Karol Strasburger, choć według wstępnych planów funkcję tę miał pełnić Paweł Wawrzecki. Program 
nagrywany jest w studiu telewizyjnym w Warszawie przy ul. Inżynierskiej 4, w budynku dawnego kina „Syrena”.
\end{minipage}
\end{figure}
W programie uczestniczą przedstawiciele dwóch drużyn, z których każda liczy 5 osób. Początkowo zespoły mogły składać się wyłącznie z 
członków jednej rodziny, lecz od 2007 roku zespoły mogą tworzyć także grupy znajomych. Każda drużyna ma wyznaczonego kapitana – głowę 
drużyny (bądź głowę rodziny). Gra polega na udzielaniu takich odpowiedzi, na które wskazało wcześniej najwięcej ankietowanych osób. 
Teleturniej dzieli się na dwa etapy: pierwszy, który trwa tak długo, aż jedna z drużyn zdobędzie łącznie przynajmniej trzysta punktów 
(składa się zatem z dowolnej liczby rund; w praktyce najczęściej pięciu, czasem czterech lub sześciu) oraz finał. Drużyna, która dostaje 
się do finału, bierze udział w następnym programie, przy czym może wystąpić w maksymalnie trzech kolejnych odcinkach. Każda drużyna ma 
swój kolor – czerwony lub niebieski; po czerwonej części studia staje drużyna, która w poprzednim odcinku weszła do finału, po 
niebieskiej zawsze drużyna debiutująca. W przypadku, gdy do gry przystępują dwie nowe drużyny, o zajmowanym stanowisku decyduje 
losowanie.

\subsection{Historyczne pory emisji}

\begin{center}
\begin{tabular}{ |c|c| }
\hline
\textbf{Przybliżony okres} & \textbf{Pora emisji}\\
\hline
sezon 1994/1995 & sobota i niedziela 16.30\\
sezon 1995/1996 & sobota i niedziela ok. 15.00\\
wakacje 1996 & piątek ok. 16.30, sobota i niedziela 15.00\\
sezon 1996/1997 & sobota i niedziela 15.\\
sezony 1997/1998 i 1998/1999 & sobota, niedziela i święta ok. 15.00\\
sezony 1999/2000 i 2000/2001 & sobota, niedziela i święta 14.30\\
od września 2001 & sobota, niedziela i święta 14.00[b]\\
\hline
\end{tabular}
\end{center}

\section{Zakończenie}
W tekście odwołuję się do różnych tematów. Najpierw jest spora ilość tekstu po łacinie. Następnie kilka przykładowych prostych wzorów na 
całki. Na samym końcu zamieściłem trochę informacji o programie Familiada. Proszę zwrócic uwagę na różnego rodzaju formatowania tekstu 
jak np. \textbf{pogrubienia} czy \underline{podkreślenia}. Mogą one być na pierwszy rzut oka mało widoczne, ze względu na długość 
tekstów. 

\section{Bibliografia}
Bibliografia do tego artykułu zgodnie z poleceniem została dodana ręcznie.\\
\\
\begin{enumerate}
\item https://www.lipsum.com/
\item https://www.overleaf.com/learn/latex/TablesTutorial
\item https://pl.wikipedia.org/wiki/Familiada
\end{enumerate}

\end{document}

